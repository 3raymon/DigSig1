\section{FIR Filtering and Convolution\buch{Chapter 4}}
\subsection{Block Processing Methods}
\subsubsection{Convolution}


\subsubsection{Direct Form}
\begin{tabular}{|l|l|}
	\hline
	$h$		& $h=[h_0,h_1, \ldots , h_M]$
	\\ \hline
	$L_h$	& $L_h = M + 1$
	\\ \hline
	$L_y$	& $L_y = L + M = L_x + L_h - 1$
	\\ \hline
	$y(n)$	& $y(n) = \sum\limits_{m=max(0,n-L+1)}^{min(n,M)} h(m)x(n-m)$
	\\ \hline 
\end{tabular}


\subsubsection{Convolution Table}


\subsubsection{LTI Form}
\[
	y(n) = \sum\limits_{m=max(0,n-M)}^{min(n,L-1)} x(m)h(n-m)
\]


\subsubsection{Matrix Form}
The convolutional eqations can also be written in the linear matrix form:
\[
	y = H  x	\qquad \text{or} \qquad y = Xh
\]
where $H$ is built out of the filter's impulse response $h$ or the signal matrix $X$ is built out of the input signal. 
The filter matrix $H$, respectivly the signal matrix $X$, must be rectangular with dimensions
\[
	L_y \times L_x = (L + M)\times L \qquad \text{or} \qquad
	L_y \times L_h = (L + M)\times (M + 1)
\]

Example:
\setArrayStretch{1}
\[
	y =
	\begin{bmatrix}
		y_0 \\
		y_1 \\
		y_3 \\
		y_4 \\
		y_5 \\
		y_6 \\
		y_7
	\end{bmatrix}
	= \begin{bmatrix}
		h_0	& 0		& 0		& 0		& 0 \\
		h_1	& h_0	& 0 	& 0 	& 0 \\
		h_2	& h_1	& h_0	& 0		& 0 \\
		h_3 & h_2	& h_1	& h_0	& 0 \\
		0	& h_3	& h_2	& h_1	& h_0 \\
		0	& 0		& h_3	& h_2	& h_1 \\
		0	& 0		& 0		& h_3	& h_2 \\
		0	& 0		& 0		& 0		& h_3	 
	  \end{bmatrix}
	  \begin{bmatrix}
	  	x_0 \\
	  	x_1 \\
	  	x_2 \\
	  	x_3 \\
	  	x_4
	  \end{bmatrix}
	= Hx
\]
\resetArrayStretch


\subsubsection{Flip-and-Slide Form}


\subsubsection{Transient and Steady-State Behavior}
\begin{tabular}{|l|l|}
	\hline
	input-on trainsients	& $ 0 \leq n < M $
	\\ \hline
	steady state			& $ M \leq n \leq L-1 $
	\\ \hline
	input-off transient		& $ L-1 < n \leq L-1+M $
	\\ \hline
\end{tabular}\newline

Therefore, the direct form takes the following different forms depending
on the value of the output index $n$:
\[
	y_n =
		\left\{
			\begin{array}{r l l l}
				\sum\limits_{m=0}^{n} & h_m x_{n-m}		& \quad \text{if } 0 \leq n < M			& \quad \text{input-on} \\
				\sum\limits_{m=0}^{M} & h_m x_{n-m}		& \quad \text{if } M \leq n \leq L-1	& \quad \text{steady state} \\
				\sum\limits_{m=n-L+1}^{M} & h_m x_{n-m}	& \quad \text{if } L-1 < n \leq L-1+M	& \quad \text{input-off}
			\end{array}
		\right.
\]


\subsubsection{Convolution of Infinite Sequences}
Three cases:
\setArrayStretch{1}
\begin{enumerate}
  \item Infinite filter, finite input; i.e., $M = \infty$, $L < \infty$
  \item Finite filter, infinite input; i.e., $M < \infty$, $L = \infty$
  \item Infinite filter, infinite input; i.e., $M = \infty$, $L = \infty$
\end{enumerate}
\resetArrayStretch

Therefore, the direct form takes the following different forms:
\[
	y_n =
		\left\{
			\begin{array}{r l l}
				\sum\limits_{m=max(0,n-L+1)}^{n} & h_m x_{n-m}		& \quad \text{if $M = \infty$, $L < \infty$} \\	
				\sum\limits_{m=0}^{min(n,M)} & h_m x_{n-m}		& \quad \text{if $M < \infty$, $L = \infty$ } \\
				\sum\limits_{m=0}^{n} & h_m x_{n-m}	& \quad \text{if $M = \infty$, $L = \infty$ }
			\end{array}
		\right.
\]




	
	