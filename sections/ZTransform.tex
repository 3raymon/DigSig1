\section{z-Transformation\buch{Chapter 5}}

\subsection{Basic Properties}

\[
	X(z) = \sum\limits_{n=-\infty}^\infty x(n)z^{-n}
\]

\begin{tabularx}{0.6\textwidth}{|l|X|}
	\hline
	linearity & $a_1x_1(n) + a_2x_2(n) \overset{Z}{\longrightarrow} a_1X_1(z) + a_2X_2(z)$
	\\ \hline
	delay	& $x(n) \overset{Z}{\longrightarrow} X(z) \Longrightarrow x(n-D) \overset{Z}{\longrightarrow} z^{-D}X(z)$
	\\ \hline
	convolution & $y(n) = h(n) \convolution x(n) \Longrightarrow Y(z) = H(z)X(z)$
	\\ \hline
\end{tabularx}


\subsection{Region of Convergence}

\[
	\{ z \in \mathbb{C} \quad | \quad X(z) = \sum\limits_{n=-\infty}^{\infty} x(n)z^{-n} \neq \pm \infty \}
\]

\begin{tabularx}{0.6\textwidth}{|l|X|}
	\hline
	infinite geometric series 1 & $1 + x + x^2 + x^3 + \ldots = \sum\limits_{n=0}^{\infty} x^n = \frac{1}{1-x}$
	\\ \hline
	infinite geometric series 2 & $x + x^2 + x^3 + \ldots = \sum\limits_{m=1}^{\infty} x^m = \frac{x}{1-x}$
	\\ \hline
\end{tabularx}


\subsection{Causality and Stability}

For a signal or system to be simultaneously stable and causal, it is necessary that all its poles lie strictly inside the unit circle in the z-plane. 
\[ 1 > \max\limits_{i}|p_i| \]

A signal or system
can also be simultaneously stable and anticausal, but in this case all its poles must lie
strictly outside the unit circle.
\[ 1 < \min\limits_{i}|p_i| \]

\subsection{Frequency Spectrum}
\begin{align*}
	X(\omega)&= \sum\limits_{n=-\infty}^{\infty} x(n)e^{-j\omega n} \quad \text{(DTFT)} \
	\qquad \omega = \frac{2 \pi f}{f_s} \qquad -\pi \leq \omega \leq \pi
\end{align*}

\begin{align*}
	x(n)&= \frac{1}{2\pi} \int\limits_{-\pi}^{\pi} X(\omega)e^{j\omega n} d\omega \quad \text{(inverse DTFT)} 
\end{align*}

Another useful relationship is Parseval’s equation, which relates the total energy of
a sequence to its spectrum:

\begin{align*}
	\sum_{n=-\infty}^{\infty} |x(n)|^2&= \frac{1}{2\pi} \int\limits_{-\pi}^{\pi} |X(\omega)|^2 d\omega \quad \text{(Parseval)} 
\end{align*}
