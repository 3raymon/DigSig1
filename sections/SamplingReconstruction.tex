\section{Sampling and Recostruction\buch{Chapter 1}}
\subsection{Analog Signals}
\begin{tabularx}{\linewidth}{|l|X|}
	\hline
	Fourier transform & $X(\Omega) = \int\limits_{-\infty}^{\infty} x(t)e^{-j\Omega t}dt \qquad \Omega = 2\pi f $ \\
	\hline
	inverse Fourier transform & $ x(t) = \int\limits_{-\infty}^{\infty} X(\Omega)e^{j\Omega t} \frac{d\Omega}{2 \pi} $ \\
	\hline
\end{tabularx}

\subsection{Digital Signals}
\subsubsection{Sampling Theorem}
Sampling means that the analog signal is periodically measured with a sampling interval T. The discrete index $n$, relates to
the time $t$ as follows:
\[ t = nT \qquad n = 0,1,2,\ldots \]
The sampling frequency relates to the sampling interval as follows:
\[ f_s = \frac{1}{T} \]
Sampling Theorem (Nyquist rate):
\[ f_s \geq 2f_{max} \qquad \text{or} \qquad T \leq \frac{1}{2f_{max}} \]
Nyquist interval:
\[ \left[-\frac{f_s}{2}, \frac{f_s}{2}\right] \]
\subsubsection{Aliasing}
If the signal frequency $f$ is outside the nyquist interval, the signal will be
aliased with $f - f_{sampling}$.\\
\textbf{Example:} $sin(8\pi t)$ (signal frequency $f=4$) sampled at a rate of
$f_s=5Hz$ will be aliased to $sin(2\pi (f-f_s) t) = sin(2\pi (-1) t)$\\

Is the signal frequency $f$ inside the nyquist interval, no aliasing will be
perceived.
